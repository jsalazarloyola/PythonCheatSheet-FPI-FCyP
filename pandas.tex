%%%%%%%%%%%%%%%%%%%%%%%%%%%%%%%%%%%%%%%%%%%%%%%%%%%%%%%%%%%%%%%%%%%%%%
% MÓDULO pandas
%%%%%%%%%%%%%%%%%%%%%%%%%%%%%%%%%%%%%%%%%%%%%%%%%%%%%%%%%%%%%%%%%%%%%%
\begin{contentbox}{label=Objetos Básicos de Pandas}
    La importación tradicional de \texttt{pandas} es
    \begin{center}
        \lstinline!import pandas as pd!
    \end{center}
    
    Un \texttt{DataFrame} es una tabla con índices \alert{ordenados} para filas y columnas. Pueden ser numerados (por defecto, posiciones de listas) o nombrados (números, strings). En ambos casos sirven como \textbf{claves}. Se crean con:
    
    \begin{lstlisting}
df = pd.DataFrame(datos, index=filas, 
                  columns=columnas,
                  dtype=tipos)
    \end{lstlisting}
    
    con los parámetros:
    \begin{description}
    \item[\texttt{datos}] matriz con los elementos de la tabla. Para listas de listas cada lista es una fila y cada elemento de ella una nueva columna.
    \item[\texttt{index}] lista con nombres de filas.
    \item[\texttt{columns}] lista con nombres de columnas.
    \item[\texttt{dtype}] Tipo de dato de las columnas
    \end{description}
\vspace{-13pt}
\end{contentbox}

\begin{contentbox}{label=DataFrame a partir de \textsc{csv}}
    \begin{lstlisting}
Tabla = pd.read_csv("ruta/fichero.csv",
                    sep=",",
                    header=n,
                    index_col=m,
                    decimal=".")
    \end{lstlisting}
    
    con los parámetros:
    \begin{description}
    \item[\texttt{sep}] - separador de columnas (por defecto \lstinline!","!).
    \item[\texttt{header}] -  número de fila del encabezado (nombres de columnas).
    \item[\texttt{index\_col}] - número de la columna con los índices de fila.
    \item[\texttt{decimal}] - Indicador de parte decimal (por defecto, \lstinline!"."!).
    \end{description}
    
    Otros parámetros:
    \begin{description}
    \item[\texttt{names}] Lista de nombres de columnas.
    \item[\texttt{usecols}] Lista de nombres de columnas a usar.
    \item[\texttt{dtype}] Diccionario de columnas y tipos.
    \end{description}
\end{contentbox}

\begin{contentbox}{label=DataFrame a partir de Hoja de Cálculo}
    \begin{lstlisting}
Tabla = pd.read_excel("ruta/fichero",
                      sheet_name=sheet)
    \end{lstlisting}
    
    Se usa \lstinline!sheet_name! para especificar el nombre o número de la hoja a cargar. Si no se hace, se carga la primera.
    
    Todos los parámetros mencionados de \lstinline!pd.read_csv! tienen el mismo significado, salvo \texttt{delimiter} y \texttt{decimal}, que no existen en esta función.
\end{contentbox}

\begin{contentbox}{label=Información de DataFrame}
    \begin{tabular}{C|p{0.65\linewidth}}
        \lstinline!df.info()! & Resumen informativo. \\
        \lstinline!df.head(n)! &  Primeras \texttt{n} líneas. \\
        \lstinline!df.tail(n)! &  Últimas \texttt{n} líneas. \\
        \lstinline!df.shape! & Tupla con las dimensiones en la forma \lstinline!(filas, columnas)!. \\
        \lstinline!df.columns! &  Iterable con nombres de columnas. \\
        \lstinline!df.index! &  Iterable con nombres de filas.
    \end{tabular}
\end{contentbox}

\begin{contentbox}{label=Acceso a un DataFrame}
    \begin{tabular}{C|p{0.55\linewidth}}
        \lstinline!df[col]! & Acceso a la columna \texttt{col}. \\
        \lstinline!df[l_cols]! & Acceso a las columnas \lstinline!l_cols! (lista de columnas). \\
        \lstinline!df[col][row]! &  Acceso al elemento en la columna \texttt{col} y la fila \texttt{row}. \\
        \lstinline!df.loc[row]! & Acceso a la fila \texttt{row}. \\
        \lstinline!df.loc[l_rows]! & Acceso a las filas \lstinline!l_rows! (lista de filas). \\
        \lstinline!df.loc[row, col]! &  Acceso al elemento en la fila \texttt{row} y la columna \texttt{row}. \\
        \lstinline!l_cols!. \\
        \lstinline!df.iloc[n]! &  Acceso a la fila \texttt{n}. \\
        \lstinline!df.iloc[n, m]! &  Acceso al elemento en la fila \texttt{n} y columna \texttt{m}. \\
    \end{tabular}
    
    Cuando accedemos a solo una fila o una columna, obtenemos una \lstinline!pd.Series!, una ``lista'' cuyos índices son las filas del \texttt{DataFrame}.
    
    La diferencia entre \texttt{iloc} y \texttt{loc} es que la primera funciona con las posiciones como enteros (equivalente a listas), mientras que la segunda, con los \alert{nombres} de las filas y columnas.
    
    Al acceder a un dato, fila o columna es posible cambiar su valor haciendo la asignación respectiva (equivalente a listas).
\end{contentbox}

\begin{contentbox}{label=Segmentos en \texttt{DataFrame}}
    La nomenclatura de segmentos puede ser utilizada en todos los mecanismos de acceso a un \texttt{DataFrame}, en cualquiera de sus dimensiones:
    
    \begin{tabular}{C|p{0.5\linewidth}}
        \lstinline!df.iloc[a:b:c]! &  Segmento de la fila \texttt{a} a \texttt{b}, sin incluirla, cada \texttt{c} filas. \\
        \lstinline!df.iloc[n, :b]! &  Fila \texttt{n} hasta la columna anterior a \texttt{b}. \\
        \lstinline!df.iloc[a:, m]! &  Desde la fila \texttt{a}, la columna \texttt{m}. \\
        \lstinline!df.iloc[a:b, c:]! &  Filas \texttt{a} hasta \texttt{b}, sin incluirla, y desde la columna \texttt{c}.
    \end{tabular}
    
    A diferencia de las posiciones enteras accedidas con \lstinline!iloc!, al utilizar los nombres de filas y columnas (con \lstinline!loc! o el operador \lstinline![]!), la columna de término sí es accedida:
    
    \begin{tabular}{C|p{0.5\linewidth}}
        \lstinline!df["c_1":"c_2"]! &  Desde la columna \lstinline!c_1! hasta \lstinline!c_2!. \\
        \lstinline!df["c_1":"c_2":2]! &  Desde la columna \lstinline!c_1! hasta \lstinline!c_2! cada dos. \\
        \lstinline!df.loc["r_1":"r_2]! &  Desde la fila \lstinline!r_1! hasta \lstinline!r_2!. \\
    \end{tabular}
\end{contentbox}

\begin{contentbox}{label=Operaciones Vectoriales}
    Las operaciones aritméticas básicas de Python son \alert{vectoriales} en \texttt{pandas}. Esto significa que, al operar dos \texttt{DataFrame} o \texttt{Series} del mismo tamaño, se operan elemento a elemento:
    \begin{lstlisting}
sumas = df["col_1"] + df["col_2"]
    \end{lstlisting}
    
    Si se operan con un escalar, este se opera sobre todos los elementos del \texttt{DataFrame} o \texttt{Series}.
    \begin{lstlisting}
amplificada = 10*df["col_1"]
    \end{lstlisting}
    
    En el caso de operaciones de comparación, resultan en un \texttt{DataFrame} o \texttt{Series} de \alert{booleanos}:
    \begin{lstlisting}
filtro = df["col_3"] > df["col_4"]
    \end{lstlisting}
    
\end{contentbox}

\begin{contentbox}{label=Operaciones Lógicas Vectoriales}
    \begin{tabular}{C|p{0.35\linewidth}}
        \lstinline!df["col_1"] & df["col_2"]! & Y lógico vectorial. \\
        \lstinline!df["col_1"] | df["col_2"]! &  O lógico vectorial. \\
        \lstinline!~df["col"]! &  No lógico vectorial. \\
        \lstinline!df["col_1"]^df["col_2"]! & O lógico excluyente vectorial.
    \end{tabular}
    
    Todas estas operaciones evalúan el operador elemento a elemento en las columnas.
\end{contentbox}

\begin{contentbox}{label=Filtrado en \texttt{pandas}}
    Alternativamente a los métodos habituales de acceso, una \texttt{Series} o \texttt{DataFrame} de booleanos puede ser válida como índice. Hacer esto genera un subconjunto de los elementos originales que contiene solo los elementos cuya posición está marcada como \lstinline!True!.
    
    Podemos usar operadores booleanos vectoriales para filtrar los datos:
    \begin{lstlisting}
# Subconjunto con los valores en 
# "col_3" entre 2 y 5
df[df["col_3"] > 2 & df["col_3"] < 5]
    \end{lstlisting}

    El resultado es un nuevo \texttt{DataFrame}.
    
    Otras operaciones booleanas (i.e. válidas como filtro) incluyen:
    
    \begin{tabular}{C|p{0.35\linewidth}}
        \lstinline!df.isin(a, b)! & Si las ubicaciones están entre \texttt{a} y \texttt{b} \\
        \lstinline!df.isna()! &  Posiciones con valores nulos o vacíos \\
        \lstinline!df.notna()! &  \lstinline!~df.isna()!. \\
    \end{tabular}
\end{contentbox}

\begin{contentbox}{label=Agregar y eliminar información}
    \begin{tabular}{C|p{0.45\linewidth}}
        \lstinline!df[Nombre] = Datos! & Agrega al final la columna \lstinline!Nombre! con los datos \lstinline!Datos!. \\
        \lstinline!df.insert(n, N, D)! & Agrega la columna \lstinline!N! con los datos \lstinline!D! en la posición \lstinline!n!. \\
        \lstinline!df.drop(Ns, axis=x)! & Elimina las filas (\lstinline!x=0!) o columnas (\lstinline!x=1!) cuyos nombres estén en (\lstinline!Ns!). \\
    \end{tabular}
    
    \vspace{\baselineskip}
    Para unir dos \lstinline!DataFrames! se puede utilizar la instrucción:
    \begin{center}
        \lstinline!new_df = pd.concat([df_1, df_2], axis=x)!    
    \end{center}
    
    donde el orden de los \lstinline!DataFrame! determina el orden en que aparecen los datos en el nuevo \lstinline!DataFrame! y el valor de \lstinline!axis! determina si la unión se hace uno sobre otro (0) o uno al lado del otro (1).
    
    Si la unión se hace lateral, los datos se ordenarán haciendo coincidir sus índices; en caso de que algunas filas no tengan los mismo índices, se generan filas con la información del \lstinline!DataFrame! original, rellenando con valores \lstinline!NaN!.
    
    En el caso de uniones verticales el parámetro opcional \lstinline!ignore_index! (\lstinline!False!, por defecto), al asignarle \lstinline!True!, permite resetear los índices.
\end{contentbox}

\begin{contentbox}{label=Exportación de Archivos}
    Los \texttt{DataFrame} pueden ser exportados a \textsc{csv} u hoja de cálculo con las siguientes funciones:
    \begin{lstlisting}
df.to_csv(fichero.csv,
          sep=",",
          header=True,
          index=True,
          decimal=".")
df.to_csv(fichero.csv,
          sheet_name="Sheet1",
          header=True,
          index=True)
    \end{lstlisting}
    donde los valores por defecto de los parámetros son los dados. Los parámetros \texttt{header} e \texttt{index} señalan si se incluyen los nombres de columnas y filas, respectivamente, en el archivo de salida.
\end{contentbox}

\begin{contentbox}{label=Agregación}
    Son \alert{funciones de agregación} aquellas que convierten varios valores en uno solo. Las siguientes funciones generan un valor, si se ejecutan sobre una columna, o una \texttt{Series}, si se ejecutan sobre varias, cuyos índices son las columnas y los valores, el resultado por columna:
    
    \begin{tabular}{C|p{0.6\linewidth}}
        \lstinline!df.count()! & Cuenta celdas no vacías \\
        \lstinline!df.max()! &  El máximo de los valores \\
        \lstinline!df.min()! & El mínimo de los valores \\
        \lstinline!df.idxmax()! &  El índice del máximo de los valores \\
        \lstinline!df.idxmin()! & El índice del mínimo de los valores \\
        \lstinline!df.sum()! & La suma de los valores de las celdas
    \end{tabular}
\end{contentbox}

\begin{contentbox}{label=Operaciones vectoriales de string}
    Una \texttt{Series} (y, por ende, una columna de un \texttt{DataFrame}) contiene el atributo \texttt{str}, que permite acceder a todas las funciones de string habituales, pero aplicadas a cada elemento de la serie. Esto incluye algunas funciones adicionales, para emular algunos operadores:
    
    \begin{tabular}{C|p{0.35\linewidth}}
        \lstinline!df["c"].str.contains(text)! & Verifica si \texttt{text} está en cada celda. \\
        \lstinline!df["c"].str.get(i)! &  Retorna el caracter en la posición \texttt{i} de cada celda.
    \end{tabular}
\end{contentbox}
