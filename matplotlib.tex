%%%%%%%%%%%%%%%%%%%%%%%%%%%%%%%%%%%%%%%%%%%%%%%%%%%%%%%%%%%%%%%%%%%%%%
% MÓDULO Matplotlib
%%%%%%%%%%%%%%%%%%%%%%%%%%%%%%%%%%%%%%%%%%%%%%%%%%%%%%%%%%%%%%%%%%%%%%
\begin{contentbox}{label=Módulo Matplotlib}
    El módulo Matplotlib permite visualizar y crear gráficos a través de su interfaz \texttt{pyplot}:
    
\begin{lstlisting}
import matplotlib.pyplot as plt
\end{lstlisting}

    Las funciones principales se obtienen a partir de este submódulo. Tras generar los gráficos, estos se visualizan usando:
\begin{lstlisting}
plt.show()
\end{lstlisting}
\end{contentbox}

\begin{contentbox}{label=Gráfico de línea}
    Se crean gráficos de líneas con \lstinline!plt.plot!. Su parámetro son arreglos o equivalentes con valores numéricos:
    
\begin{lstlisting}
# Gráfico de y versus sus índices
g = plt.plot(y)
# Gráfico de y versus x
g = plt.plot(x, y)
# Gráfico de y versus x rojo
g = plt.plot(x, y, 'r')
\end{lstlisting}
    
    Retorna una lista de gráficos.
    
    Pueden graficarse varias líneas agrupando dos o tres parámetros: eje $x$, eje $y$, formato (opcional):
\begin{lstlisting}
g = plt.plot(x_1, y, x_1, z, x_2, f)
\end{lstlisting}
\end{contentbox}

\begin{contentbox}{label=Estilo de las líneas}
    El estilo del gráfico se controla mediante la función \lstinline!plt.setp! y sus parámetros:
\begin{lstlisting}
plt.setp(g, param_1=valor,
         param_2=valor,...)
\end{lstlisting}
    El parámetro \lstinline!g! es un gráfico o lista de gráficos. El resto de parámetros incluye:
    
    \begin{tabular}{C|p{0.5\textwidth}}
        color & Color de la línea. \\
        marker & Tipo de marcador. \\
        markersize & Tamaño de marcador en puntos. \\
        linestyle & Tipo de línea. \\
        linewidth & Ancho de línea en puntos.
    \end{tabular}

\end{contentbox}

\begin{contentbox}{label=Gráfico Circular}
    Se genera con \lstinline!plt.pie! y requiere un vector o lista con valores. Genera una ``torta'' de radio 1. Sus parámetros incluyen:

    \begin{tabular}{C|p{0.5\textwidth}}
        autopct & \lstinline!str! para etiquetar las porciones con su valor. \\
        labels & Secuencia de etiquetas. \\
        explode & Secuencia con distancia del centro para cada porción. \\
        labeldistance & Distancia de la etiqueta. \\
        rotatelabels & Para \lstinline!True!, rota las etiquetas para cada porción.
    \end{tabular}
    
    Ejemplo:
\begin{lstlisting}
valores = [0, 2, 1, 3]
etiquetas = ["a", "b", "c", "d"]
plt.pie(valores, labels=etiquetas,
        autopct="%1.f%%")
\end{lstlisting}
\end{contentbox}

\begin{contentbox}{label=Histograma}
    Muestra frecuencia de ocurrencia de valores en contenedores (\textit{bins}). Su parámetro \lstinline!bins! es cuántos contenedores hay o una secuencia que los especifique:
    
\begin{lstlisting}
dados = np.random.randint(1, 10, 10000)
plt.hist(dados, bins=5)
\end{lstlisting}

\end{contentbox}

\begin{contentbox}{label=Gráfico de Barras}
    Un gráfico de barras se construye igual que el de líneas, pero solo uno a la vez. Sus parámetros principales son posición y alto de cada barra:
\begin{lstlisting}
plt.bar(np.arange(len(valores)),
        valores, width=0.8,
        align='center')
\end{lstlisting}

    Los valores por defecto de los parámetros son los dados.
    
    Para un gráfico de barras horizontal, se usa \lstinline!plt.barh! y se intercambian los parámetros height y width:
\begin{lstlisting}
plt.barh(np.arange(len(valores)),
         valores, height=0.8,
         align='center')
\end{lstlisting}
\end{contentbox}

\begin{contentbox}{label=Etiquetas del gráfico}
\begin{lstlisting}
plt.title("Título del gráfico")
plt.xlabel("Eje $x$")
plt.ylabel("Eje $y$")
# Posiciones y etiquetas deben
# ser del mismo largo
# Etiquetas pueden ser strings
plt.xticks(posiciones_x, etiquetas_x)
plt.yticks(posiciones_y, etiquetas_y)
\end{lstlisting}
\end{contentbox}

\begin{contentbox}{label=Leyenda}
    Las leyendas se incluyen a partir de las etiquetas de cada gráfico (parámetro opcional \lstinline!label!) o directamente con \lstinline!plt.legend!.
    
\begin{lstlisting}
# Genera leyenda a partir de etiquetas
# previas
plt.legend()
# Explícitamente se dan las leyendas
# (en orden de creación)
plt.legend(["a", "b", "c"])
# Asignando explícitamente a cada uno
plt.legend(g, "a")
\end{lstlisting}
\end{contentbox}

\begin{contentbox}{label=Múltiples gráficos}
\begin{lstlisting}
# Crea y selecciona figura en blanco
fig = plt.figure()
# Selecciona figura n
fig = plt.figure(n)
# Selecciona el subgráfico i
# al subdividir la figura en n por m
plt.subplot(n, m, i)
\end{lstlisting}
    Se crean los gráficos en la última figura seleccionada.
\begin{lstlisting}
# Crea figura y un eje para graficar
fig, ax = plt.subplots()
ax.plot(x, y)
# Crea figura y arreglo de n ejes
# para graficar
fig, ax = plt.subplots(n)
ax[0].plot(x, y)
# Crea figura y matriz de
# n por m de ejes
fig, ax = plt.subplots(n, m)
ax[0, 0].plot(x, y)
\end{lstlisting}
\end{contentbox}
