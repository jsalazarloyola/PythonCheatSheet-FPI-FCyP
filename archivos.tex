% !TEX root=main.tex
%%%%%%%%%%%%%%%%%%%%%%%%%%%%%%%%%%%%%%%%%%%%%%%%%%%%%%%%%%%%%%%%%%%%%%
% ARCHIVOS
%%%%%%%%%%%%%%%%%%%%%%%%%%%%%%%%%%%%%%%%%%%%%%%%%%%%%%%%%%%%%%%%%%%%%%

\begin{contentbox}{label=Archivos}
    Un archivo es un objeto especial en Python creado con la función nativa \lstinline!open!:
    
    \begin{center}
        \lstinline!file = open(ruta, modo)!
    \end{center}
    
    El parámetro \texttt{ruta} (\lstinline!str!) especifica dónde encontrar el archivo (carpetas y nombre).
    
    Cuando dejamos de usar el archivo, debemos cerrarlo, para lo que se utiliza el método \lstinline!close!, como en \lstinline!file.close()!.
\end{contentbox}
    
\begin{contentbox}{label=Parámetros de archivo}
    Pueden darse tres modos de apertura:
    
    \begin{center}
        \begin{tabular}{C|p{0.5\textwidth}}
            \textnormal{Opción} & Modo \\
            \hline
            r & Lectura \\
            w & Escritura \\
            a & Añadir
        \end{tabular}
    \end{center}
    
    Si la codificación del archivo no es la del sistema, se puede especificar con \lstinline!encoding!:
    \begin{lstlisting}
file = open(ruta, modo,
            encoding="utf8")
    \end{lstlisting}
    
    Las principales codificaciones aquí son \lstinline!utf8! (Linux, MacOS) y \lstinline!latin1! (Windows).

    Si no especifica modo, asume lectura y si no especifica codificación, asume la del sistema.
\end{contentbox}
    
\begin{contentbox}{label=Operación \lstinline!with!}
    Podemos utilizar un administrador de contexto para trabajar con archivos, a través del bloque \lstinline!with ... as...!, en cuyo caso no es necesario utilizar \lstinline!close!:
    \begin{lstlisting}
with open(ruta, modo) as file:
    operaciones_sobre(file)
    \end{lstlisting}
\end{contentbox}

\begin{contentbox}{label=Operaciones sobre Archivos}
    Los métodos básicos de archivo son:
    
    \begin{tabular}{C|p{0.5\linewidth}}
        \lstinline!file = open(r, m)! & Abre el archivo en la ubicación \texttt{r} con modo \texttt{m}. \\
        \lstinline!file.read()! &  \multirow{2}{=}{Lee todo el archivo o hasta \texttt{n} caracteres.} \\
        \lstinline!file.read(n)! &  \\
        \lstinline!file.readline()! &  Lee como string hasta el próximo final de línea. \\
        \lstinline!file.readlines()! &  Lee como lista de strings todo el archivo. \\
        \lstinline!file.write(s)! & Escribe el string \texttt{s} en el archivo. \\
        \lstinline!file.writelines(l)! & Escribe la lista de strings \texttt{l} en el archivo. \\
        \lstinline!file.close()! & Cierra el archivo, impidiendo futuros accesos. \\
    \end{tabular}
    
    Cada lectura y escritura se hacen a partir de donde terminó la anterior. Toda lectura y escritura es \emph{literal}: no se añaden ni quitan caracteres.

    Un archivo, además, puede leerse utilizando ciclos \lstinline!for!.
\end{contentbox}

\begin{contentbox}{label=Parámetros de \lstinline!print!}
    La función \lstinline!print! tiene tres parámetros por palabra clave de interés:
    \begin{itemize}
        \item \lstinline!sep=' '! el separador de elementos.
        \item \lstinline!end='\n'! el caracter final a imprimir.
        \item \lstinline!file=sys.stdout! el archivo donde imprimir (por defecto es la consola).
    \end{itemize}
\end{contentbox}
