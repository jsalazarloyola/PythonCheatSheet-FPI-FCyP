%%%%%%%%%%%%%%%%%%%%%%%%%%%%%%%%%%%%%%%%%%%%%%%%%%%%%%%%%%%%%%%%%%%%%%
% ARCHIVOS
%%%%%%%%%%%%%%%%%%%%%%%%%%%%%%%%%%%%%%%%%%%%%%%%%%%%%%%%%%%%%%%%%%%%%%

\begin{contentbox}{label=Ciclos con \texttt{for}}
    Repite la acción por cada elemento de la secuencia:
\begin{lstlisting}
for elemento in secuencia:
    acciones_a_repetir
\end{lstlisting}
    
    La variable \lstinline!elemento! es definida en está secuencia y su valor es cada elemento de la secuencia en orden.
    
    Entre las secuencias, se incluyen (entre muchos otros) archivos, \lstinline!list!, \lstinline!str!, \lstinline!range!, etc.
\end{contentbox}

\begin{contentbox}{label=Función \lstinline!range!}
    Generación de secuencias numéricas:
    \begin{itemize}
        \item \lstinline!range(stop)! genera números de 0 a \texttt{stop-1}.
        \item \lstinline!range(start, stop, d)! genera números desde \texttt{start} hasta \texttt{stop-1}, con distancia de \texttt{d} entre ellos.
    \end{itemize}
    
%     Por ejemplo, para mostrar todos los números desde 0 a $n-1$:
% \begin{lstlisting}
% for i in range(n):
%     print(i)
% \end{lstlisting}

    Nótese que \lstinline!range! no es una lista, pero puede convertirse a una mediante \lstinline!list(range(n))!.
\end{contentbox}

\begin{contentbox}{label=Archivos}
    Un archivo es un objeto especial en Python creado con la función nativa \lstinline!open!:
    
    \begin{center}
        \lstinline!file = open(ruta, modo)!
    \end{center}
    
    El parámetro \texttt{ruta} (\lstinline!str!) especifica dónde encontrar el archivo (carpetas y nombre).
    
    Cuando dejamos de usar el archivo, debemos cerrarlo, para lo que se utiliza el método \lstinline!close!, como en \lstinline!file.close()!.
\end{contentbox}
    
\begin{contentbox}{label=Parámetros de archivo}
    Pueden darse tres modos de apertura:
    
    \begin{center}
        \begin{tabular}{C|p{0.5\textwidth}}
            \textnormal{Opción} & Modo \\
            \hline
            r & Lectura \\
            w & Escritura \\
            a & Añadir
        \end{tabular}
    \end{center}
    
    Si la codificación del archivo no es la del sistema, se puede especificar con \lstinline!encoding!:
    \begin{lstlisting}
file = open(ruta, modo,
            encoding="utf8")
    \end{lstlisting}
    
    Las principales codificaciones aquí son \lstinline!utf8! y \lstinline!latin1!.
\end{contentbox}
    
\begin{contentbox}{label=Operación \lstinline!with!}
    Podemos utilizar un administrador de contexto para trabajar con archivos, a través del bloque \lstinline!with ... as...!, en cuyo caso no es necesario utilizar \lstinline!close!:
    \begin{lstlisting}
with open(ruta, modo) as file:
    operaciones_sobre(file)
    \end{lstlisting}
\end{contentbox}

\begin{contentbox}{label=Operaciones sobre Archivos}
    Los métodos básicos de archivo son:
    
    \begin{tabular}{C|p{0.5\linewidth}}
        \lstinline!file = open(r, m)! & Abre el archivo en la ubicación \texttt{r} con modo \texttt{m}. \\
        \lstinline!file.read()! &  \multirow{2}{=}{Lee todo el archivo o hasta \texttt{n} caracteres.} \\
        \lstinline!file.read(n)! &  \\
        \lstinline!file.readline()! &  Lee como string hasta el próximo final de línea. \\
        \lstinline!file.readlines()! &  Lee como lista de strings todo el archivo. \\
        \lstinline!file.write(s)! & Escribe el string \texttt{s} en el archivo. \\
        \lstinline!file.writelines(l)! & Escribe la lista de strings \texttt{l} en el archivo. \\
        \lstinline!file.close()! & Cierra el archivo, impidiendo futuros accesos. \\
    \end{tabular}
    
    Cada lectura y escritura se hacen a partir de donde terminó la anterior. Toda lectura y escritura es \emph{literal}: no se añaden ni quitan caracteres.
    
%     Además, el archivo puede ser leído como una secuencia mediante \lstinline!for!, en cuyo caso cada elemento de la secuencia es una línea (string):
% \begin{lstlisting}
% for line in file:
%     print(line.strip("\n"))
% \end{lstlisting}
% \vspace{-12pt}
\end{contentbox}
